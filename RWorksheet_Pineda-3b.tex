% Options for packages loaded elsewhere
\PassOptionsToPackage{unicode}{hyperref}
\PassOptionsToPackage{hyphens}{url}
%
\documentclass[
]{article}
\usepackage{amsmath,amssymb}
\usepackage{iftex}
\ifPDFTeX
  \usepackage[T1]{fontenc}
  \usepackage[utf8]{inputenc}
  \usepackage{textcomp} % provide euro and other symbols
\else % if luatex or xetex
  \usepackage{unicode-math} % this also loads fontspec
  \defaultfontfeatures{Scale=MatchLowercase}
  \defaultfontfeatures[\rmfamily]{Ligatures=TeX,Scale=1}
\fi
\usepackage{lmodern}
\ifPDFTeX\else
  % xetex/luatex font selection
\fi
% Use upquote if available, for straight quotes in verbatim environments
\IfFileExists{upquote.sty}{\usepackage{upquote}}{}
\IfFileExists{microtype.sty}{% use microtype if available
  \usepackage[]{microtype}
  \UseMicrotypeSet[protrusion]{basicmath} % disable protrusion for tt fonts
}{}
\makeatletter
\@ifundefined{KOMAClassName}{% if non-KOMA class
  \IfFileExists{parskip.sty}{%
    \usepackage{parskip}
  }{% else
    \setlength{\parindent}{0pt}
    \setlength{\parskip}{6pt plus 2pt minus 1pt}}
}{% if KOMA class
  \KOMAoptions{parskip=half}}
\makeatother
\usepackage{xcolor}
\usepackage[margin=1in]{geometry}
\usepackage{color}
\usepackage{fancyvrb}
\newcommand{\VerbBar}{|}
\newcommand{\VERB}{\Verb[commandchars=\\\{\}]}
\DefineVerbatimEnvironment{Highlighting}{Verbatim}{commandchars=\\\{\}}
% Add ',fontsize=\small' for more characters per line
\usepackage{framed}
\definecolor{shadecolor}{RGB}{248,248,248}
\newenvironment{Shaded}{\begin{snugshade}}{\end{snugshade}}
\newcommand{\AlertTok}[1]{\textcolor[rgb]{0.94,0.16,0.16}{#1}}
\newcommand{\AnnotationTok}[1]{\textcolor[rgb]{0.56,0.35,0.01}{\textbf{\textit{#1}}}}
\newcommand{\AttributeTok}[1]{\textcolor[rgb]{0.13,0.29,0.53}{#1}}
\newcommand{\BaseNTok}[1]{\textcolor[rgb]{0.00,0.00,0.81}{#1}}
\newcommand{\BuiltInTok}[1]{#1}
\newcommand{\CharTok}[1]{\textcolor[rgb]{0.31,0.60,0.02}{#1}}
\newcommand{\CommentTok}[1]{\textcolor[rgb]{0.56,0.35,0.01}{\textit{#1}}}
\newcommand{\CommentVarTok}[1]{\textcolor[rgb]{0.56,0.35,0.01}{\textbf{\textit{#1}}}}
\newcommand{\ConstantTok}[1]{\textcolor[rgb]{0.56,0.35,0.01}{#1}}
\newcommand{\ControlFlowTok}[1]{\textcolor[rgb]{0.13,0.29,0.53}{\textbf{#1}}}
\newcommand{\DataTypeTok}[1]{\textcolor[rgb]{0.13,0.29,0.53}{#1}}
\newcommand{\DecValTok}[1]{\textcolor[rgb]{0.00,0.00,0.81}{#1}}
\newcommand{\DocumentationTok}[1]{\textcolor[rgb]{0.56,0.35,0.01}{\textbf{\textit{#1}}}}
\newcommand{\ErrorTok}[1]{\textcolor[rgb]{0.64,0.00,0.00}{\textbf{#1}}}
\newcommand{\ExtensionTok}[1]{#1}
\newcommand{\FloatTok}[1]{\textcolor[rgb]{0.00,0.00,0.81}{#1}}
\newcommand{\FunctionTok}[1]{\textcolor[rgb]{0.13,0.29,0.53}{\textbf{#1}}}
\newcommand{\ImportTok}[1]{#1}
\newcommand{\InformationTok}[1]{\textcolor[rgb]{0.56,0.35,0.01}{\textbf{\textit{#1}}}}
\newcommand{\KeywordTok}[1]{\textcolor[rgb]{0.13,0.29,0.53}{\textbf{#1}}}
\newcommand{\NormalTok}[1]{#1}
\newcommand{\OperatorTok}[1]{\textcolor[rgb]{0.81,0.36,0.00}{\textbf{#1}}}
\newcommand{\OtherTok}[1]{\textcolor[rgb]{0.56,0.35,0.01}{#1}}
\newcommand{\PreprocessorTok}[1]{\textcolor[rgb]{0.56,0.35,0.01}{\textit{#1}}}
\newcommand{\RegionMarkerTok}[1]{#1}
\newcommand{\SpecialCharTok}[1]{\textcolor[rgb]{0.81,0.36,0.00}{\textbf{#1}}}
\newcommand{\SpecialStringTok}[1]{\textcolor[rgb]{0.31,0.60,0.02}{#1}}
\newcommand{\StringTok}[1]{\textcolor[rgb]{0.31,0.60,0.02}{#1}}
\newcommand{\VariableTok}[1]{\textcolor[rgb]{0.00,0.00,0.00}{#1}}
\newcommand{\VerbatimStringTok}[1]{\textcolor[rgb]{0.31,0.60,0.02}{#1}}
\newcommand{\WarningTok}[1]{\textcolor[rgb]{0.56,0.35,0.01}{\textbf{\textit{#1}}}}
\usepackage{graphicx}
\makeatletter
\def\maxwidth{\ifdim\Gin@nat@width>\linewidth\linewidth\else\Gin@nat@width\fi}
\def\maxheight{\ifdim\Gin@nat@height>\textheight\textheight\else\Gin@nat@height\fi}
\makeatother
% Scale images if necessary, so that they will not overflow the page
% margins by default, and it is still possible to overwrite the defaults
% using explicit options in \includegraphics[width, height, ...]{}
\setkeys{Gin}{width=\maxwidth,height=\maxheight,keepaspectratio}
% Set default figure placement to htbp
\makeatletter
\def\fps@figure{htbp}
\makeatother
\setlength{\emergencystretch}{3em} % prevent overfull lines
\providecommand{\tightlist}{%
  \setlength{\itemsep}{0pt}\setlength{\parskip}{0pt}}
\setcounter{secnumdepth}{-\maxdimen} % remove section numbering
\ifLuaTeX
  \usepackage{selnolig}  % disable illegal ligatures
\fi
\IfFileExists{bookmark.sty}{\usepackage{bookmark}}{\usepackage{hyperref}}
\IfFileExists{xurl.sty}{\usepackage{xurl}}{} % add URL line breaks if available
\urlstyle{same}
\hypersetup{
  pdftitle={RWorksheet\_Pineda\#3b.Rmd},
  hidelinks,
  pdfcreator={LaTeX via pandoc}}

\title{RWorksheet\_Pineda\#3b.Rmd}
\author{}
\date{\vspace{-2.5em}2023-10-11}

\begin{document}
\maketitle

\hypertarget{r-markdown}{%
\subsection{R Markdown}\label{r-markdown}}

This is an R Markdown document. Markdown is a simple formatting syntax
for authoring HTML, PDF, and MS Word documents. For more details on
using R Markdown see \url{http://rmarkdown.rstudio.com}.

When you click the \textbf{Knit} button a document will be generated
that includes both content as well as the output of any embedded R code
chunks within the document. You can embed an R code chunk like this:

\begin{Shaded}
\begin{Highlighting}[]
\FunctionTok{summary}\NormalTok{(cars)}
\end{Highlighting}
\end{Shaded}

\begin{verbatim}
##      speed           dist       
##  Min.   : 4.0   Min.   :  2.00  
##  1st Qu.:12.0   1st Qu.: 26.00  
##  Median :15.0   Median : 36.00  
##  Mean   :15.4   Mean   : 42.98  
##  3rd Qu.:19.0   3rd Qu.: 56.00  
##  Max.   :25.0   Max.   :120.00
\end{verbatim}

\hypertarget{including-plots}{%
\subsection{Including Plots}\label{including-plots}}

You can also embed plots, for example:

\includegraphics{RWorksheet_Pineda-3b_files/figure-latex/pressure-1.pdf}

Note that the \texttt{echo\ =\ FALSE} parameter was added to the code
chunk to prevent printing of the R code that generated the plot.

\#1.a Respondents\_Num \textless- c(1:20) Sex \textless-
c(2,2,1,2,2,2,2,2,2,2,1,2,2,2,2,2,2,2,1,2) Father\_Occ \textless-
c(1,3,3,3,1,2,3,1,1,1,3,2,1,3,3,1,3,1,2,1) Person\_Home \textless-
c(5,7,3,8,5,9,6,7,8,4,7,5,4,7,8,8,3,11,7,6) Siblings\_Sch \textless-
c(6,4,4,1,2,1,5,3,1,2,3,2,5,5,2,1,2,5,3,2) House\_Type \textless-
c(1,2,3,1,1,3,3,1,2,3,2,3,2,2,3,3,3,3,3,2)

Data\_Household \textless- data.frame(Respondents = Respondents\_Num,
Sex = Sex, FathersOccupation = Father\_Occ, PersonAtHome = Person\_Home,
SiblingsAtSchool = Siblings\_Sch, HouseType = House\_Type)
Data\_Household

\#1.b

str(Data\_Household) summary(Data\_Household)

\#The data frame consists of 20 observations(rows) and 6 variables
(columns) \#The data frame consists of 6 variables (columns) and 20
observations (rows)

\#Respondents - which contains a numeric identifier for each repondents
\#Sex - (1 for male, 2 for female) represents the respondent's gender
\#Fathers Occupation - indicates the father's occupaation ( 1 for
farmer, 2 for driver, 3 for others) \#Person at Home - shows number of
person at home \#Siblings at School - number of siblings at school
\#Type of House - describes the type of house (1 for wood, 2 for
semi-concrete, 3 for concrete)

\#1.c Siblings\_Sch\_mean \textless-
mean(Data\_Household\$SiblingsAtSchool) Siblings\_Sch\_mean

\#The mean of SIblings at School is 2.95

\#1.d First\_TwoRows \textless- Data\_Household{[}1:2,{]} First\_TwoRows

\#1.e ThirdandForth\_Rows \textless- Data\_Household{[}c(3,5),c(2,4){]}
ThirdandForth\_Rows

\#1.f House\_Type \textless- Data\_Household\$HouseType House\_Type

\#1.g MaleFarmer \textless-
Data\_Household{[}Data\_Household\(Sex == 1 & Data_Household\)FathersOccupation
== 1,{]} MaleFarmer

\#1.h FemaleRespondents \textless-
Data\_Household{[}Data\_Household\$SiblingsAtSchool \textgreater=5,{]}
FemaleRespondents

\#There are 5 observations

\#2 Data\_Frame =
data.frame(Ints=integer(),Doubles=double(),Characters=character(),Logicals=logical(),Factors=factor(),stringsAsFactors=FALSE)

print(``Structure of the Empty Data Frame:'') print(str(Data\_Frame))

\#Data\_Frame have empty date frame with 0 rows and 5 columns \#The
columns have the following date type: \#Int = integer \#Doubles = double
\#Characters = character \#Logicals = logical \#Factors = factor \#0
means leves are empty \#Can be served as template where it can be
populated with data

\#3 RespondentsNew \textless- c(1:10) SexNew \textless- c(``Male'',
``Female'', ``Female'', ``Male'', ``Male'', ``Female'', ``Female'',
``Male'', ``Female'', ``Male'') OccNew \textless- c(1,2,3,3,1,2,2,3,1,3)
PersonHome\_New \textless- c(5,7,3,8,6,4,4,2,11,6) SibsNew \textless-
c(2,3,0,5,2,3,1,2,6,2) HouseNew \textless- c(``Wood'', ``Congrete'',
``Congrete'', ``Wood'', ``Semi-Congrete'', ``Semi-Congrete'', ``Wood'',
'' Semi-Congrete'',``Semi-Congrete'', ``Congrete'')

Data\_Household \textless- data.frame(Respondents = RespondentsNew, Sex
= SexNew, FathersOccupation = OccNew, PersonAtHome = PersonHome\_New,
SiblingsAtSchool = SibsNew, HouseType = HouseNew)

write.csv(Data\_Household, file = ``DataHousehold.csv'')

\#3.a imported \textless- read.csv(``DataHousehold.csv'') imported

\#3.b imported\(Sex <- factor(imported\)Sex, levels = c(``Male'',
``Female'')) imported\(Sex <- as.integer(imported\)Sex) imported\$Sex

\#3.c imported\(HouseType <- factor(imported\)HouseType, levels =
c(``Wood'', ``Congrete'', ``Semi-Congrete''))
imported\(HouseType <- as.integer(imported\)HouseType)
imported\$HouseType

\#3.d imported\(FathersOccupation <- factor(imported\)FathersOccupation,
levels = c(1,2,3), labels = c(``Farmer'', ``Driver'', ``Others''))
imported\(FathersOccupation <- as.integer(imported\)FathersOccupation)
imported\$FathersOccupation

\#3.e FemaleDriver \textless-
(imported\(Sex == 2 & imported\)FathersOccupation == ``Driver'')
FemaleDriver

\#3.f Greater\_Five \textless- imported{[}imported\$SIblingsAtSchool
\textgreater=5,{]} Greater\_Five

\#4 \#On this day, July 14th, negative sentiments outnumber all other
sentiments. This signifies that some topics or occurrences got negative
sentiments on that particular day.

\#Even though all sentiments increased on this day, July 15th, negative
sentiments remained at their peak. This implies that something happened
on that day that triggered unfavorable feelings.

\#On these days, July 17th and 18th, negative attitudes are high,
whereas neutral and positive sentiments are similar.

\#On July 20th, all sentiments reached their lowest point, albeit there
were still more negative sentiments among the others. This demonstrates
that nothing happened that day.

\#On this day, July 21st, all sentiments rise, with the negative
dominating. This could imply that something happened that day that
prompted negative responses from the people.

\#We can conclude from this data that public sentiment is open to
external forces and that it evolves over time.

\end{document}
